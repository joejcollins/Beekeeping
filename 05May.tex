% !TeX root = 00beekeeping.tex
\section{May: Increase to Four Colonies}

Swarming is most likely in May so 
it is an opportunity to raise two new queens.
The intention is to 
begin the month with two colonies on 18 frames each (9Fx9F)
and 
end the month (or June) with four colonies on 11 frames each (11F).

\begin{apiary}{Begin with two colonies (and two empty hives)}
    \path (0,0)  pic{stand};
    
    \path (4,6) pic{roof};
    \path (4,4)  pic{brood=9F};
    \path (4,2)  pic{brood=9F};
    \path (4,0)  pic{stand};
    
    \path (8,4) pic{roof};
    \path (8,2) pic{brood=2D};
    \path (8,0) pic{stand};

    \path (16,0)  pic{stand};
    
    \path (20,6) pic{roof};
    \path (20,4) pic{brood=9F};
    \path (20,2) pic{brood=9F};
    \path (20,0) pic{stand};
    
    \path (24,4) pic{roof};
    \path (24,2) pic{brood=2D};
    \path (24,0) pic{stand};
\end{apiary}

Beside each of the colonies is an empty hive containing two dummy boards.
When swarming begins the old queen will be removed to the empty hive
and
a new queen will be raised in the original hive.
There is an empty hive stand to the other side of each colony
to allow flying bees to be bled off from the old queen's colony
by swapping the hive to the other side of the new queen.

\subsection{Swarming Inspections}

Capable of making a queen, so egg in cell not only grub.

5 to 6 day interval

Strategy

Find Q first,
Put aside you are doing Celia
Then find QCs and mark
Then shuffle frames

Find QC first
Put aside, you are going artificial swarm
Mark QC
Don’t find Q then you are guessing
 
When you do an artificial swarm transfer a single frame of brood.  
Not to “keep her there” but so you can confirm that the queen is still present because she will most likely be on this frame.


\subsection{Swarm Control - Remove the Queen}

History as a guide.

\begin{table}[H]%
\begin{center}
\begin{tabular}{lllcc}
\textbf{Day} & \textbf{Queen} & \textbf{Early Beekeeper} & \textbf{Late Beekeeper} \\
1 & \rdelim\}{3}{3mm}[\textsf{Egg}] & $\leftarrow$ 1. \textbf{Remove queen} \\
2 & & \\
3 & \multirow{2}{*}{\quad $\leftarrow$ Hatching} & \\
\cline{1-1}
4 & \rdelim\}{5}{3mm}[\textsf{Larva}] &  \\
5 \\
6 \\
7 \\
8 & \multirow{2}{*}{\quad $\leftarrow$ Sealing} & $\leftarrow$ 2. \textbf{Cull queen cells} & $\leftarrow$ 1. \textbf{Remove queen} \\
\cline{1-1}
9 & \rdelim\}{8}{3mm}[\textsf{Pupa}] &  \\
10 \\
11 \\
12 \\
13 \\
14 \\
15 & & & $\leftarrow$ 2. \textbf{Cull queen cells} \\
16 & \multirow{2}{*}{\quad $\leftarrow$ Emerging} \\
\cline{1-1}
17 & \rdelim\}{5}{3mm}[\textsf{Maturing}] \\
18 \\
19 \\
20 & & \multicolumn{2}{l}{$\leftarrow$  3. \textbf{Add eggs and brood from original queen}} \\
21 \\
\cline{1-1}
22 & \rdelim\}{4}{3mm}[\textsf{Mating (typical)}] \\
23 \\
24 \\
25 \\
\cline{1-1}
26 & \rdelim\}{2}{3mm}[\textsf{Sperm Transfer}] \\
27 & & \multicolumn{2}{l}{$\leftarrow$  4. \textbf{Check and add more eggs and brood}} \\
\cline{1-1}
28 &   \rdelim\}{7}{3mm}[\textsf{Laying (typical)}] \\
29 \\
30 \\
31 \\
31 \\
33 \\
34  & & \multicolumn{2}{l}{$\leftarrow$  5. \textbf{Check (and maybe add more eggs and brood)}} \\
\end{tabular}
\caption{Swarm Control and Queen Raising}%
\end{center}
\end{table}

Take out the queen to one side along with 7 of the least brood laden frames.  
Add 4 new frames.


Put in a brood box with two dummy boards.
We want the brood to hatch in the production hive.
The queen to one side will be the feeder hive.

The other 11 frames with the queenlesshive

\begin{enumerate}

\item \textbf{Remove queen}
As soon as occupied queen cells are discovered (eggs or grubs)

Not trying to work out if they are going to swarm.

\begin{description}
  \item[Find queen and remove her], on a frame of (mostly) sealed brood + bees. Remove any queen cells from this frame after checking that there are others in the hive.

If you can't find her then guess.

Put old queen on a stand right beside the stand with the new queen.
So if there is a problem the two colonies can be united.

  \item[Add six other frames] least brood.
  
Put frame + queen in nucleus.
Add a second frame of mostly sealed brood, if wished + a frame of food + another l or 2
frames of comb (preferably) or foundation.
Shake in sufficient young worker bees to ensure that there are enough to cover the brood.
Close up the nucleus. Put green grass in the entrance if it is to remain in the same apiary.
Check through the parent colony.  Mark frames containing 2 or 3 good, unsealed, queen cells with a drawing pin.
Close up the frames in the brood box and till the remaining space with frames of comb or foundation.
Remove any sealed queen cells (Although, to use this method, the old queen must still be present so there should not be any.)
 \end{description}
 
 
\item \textbf{Cull queen cells}
7 days after the queen was removed.

\begin{description}
  \item[Remove any emergency queen cells] Be maticulous.  Go through it twice.
  
  
Go through parent colony and remove any emergency queen cells.  Best to to this three times so every other day.  The last check being a week

  \item[Keep one (only one) queen cell]
Select l of the cells previously marked and remove the others.
Some authorities suggest leaving two, incase one is a dud.
More than likely you will get a swarm.
In the event that the one left is a dud the old queen is still available and laying to provide eggs for an emergency queens,
or to unite back with the colony.

  \item[Add four frames for old queen] The old colony should be building up, add 4 more frames and remove the dummy boards to bring it up to 11 frames.
  \item[Move the old queens colony] to the other side to bleed off bees into the new queens colony.
\end{description}

Advantages of the method
Colony remains strong throughout.
Old queen is kept safe and is available if the new queen does not succeed.
The old queen in the nucleus quickly comes back into lay and her brood can be put back into the parent colony.
The method involves minimum time and lifting.
The nucleus is available to use for other procedures later, or can be united back to the original colony.

\item \textbf{Add eggs and brood from original queen (optional)}

Some authorities suggest that i
Queen isn't mated and there is increased risk you will squish her or interfere with the mating fly.
However it is an opportunity to keep the number of young bees up.
and will give early warning if the queen was injured.

\begin{description}
  \item[Confirm queen has emerged] look for chewed open.
  \item[Transfer 2 frames between] Brood and eggs to keep strong and to check for
  \item[Restack Supers] so the heavy ones are on top.
  \item[Swap Original hive] to the other side.
\end{description}

\item \textbf{Check and add more eggs and brood}

\begin{description}
  \item[Check transfered frame] look for emergency cells.
\end{description}

\item \textbf{Check (and maybe add more eggs and brood)}

\begin{description}
  \item[Check queen is laying] 
  \item[Add more eggs and brood] to check queen and if need to deplete origin
\end{description}

\end{enumerate}




