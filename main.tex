\documentclass[12pt, a5paper, twoside]{book}
\usepackage[utf8]{inputenc}
\usepackage[english]{babel}
\usepackage[margin=2.5cm]{geometry} % Easier control over the margins
\usepackage{fancyhdr}
\usepackage{titlesec}
\titleformat{\chapter}[display]
  {\normalfont\bfseries}{}{0pt}{\Huge}

\newcommand\Chapter[2]{\chapter
  [#1\hfil\hbox{}\protect\linebreak{\itshape#2}]%
  {#1\\[2ex]\Large\itshape#2}%
  \markboth{\MakeUppercase{\chaptername\ \thechapter.\ #1}}{}%
}

\renewcommand{\familydefault}{cmss}
\newcommand{\smaller}{\small}
\raggedright 

\title{Beekeeping Notes}
\author{Joe J Collins}
\date{ }
 
\begin{document}
 
\maketitle


 
\Chapter{April}{Two colonies to begin the season}

\section{1st inspection - 6 April}

\begin{enumerate}
	\item{Queen present and laying}
\end{enumerate}

\section{Marking and Clipping Queens}

\section{Select for cull or sale}



\Chapter{May}{Increase to 4 colonies}

\section{Swarm inspections}

5 day interval

Strategy

Find Q first,
Put aside you are doing Celia
Then find QCs and mark
Then shuffle frames

Find QC first
Put aside, you are going artificial swarm
Mark QC
Don’t find Q then you are guessing
 
When you do an artificial swarm transfer a single frame of brood.  Not to “keep her there” but so you can confirm that the queen is still present because she will most likely be on this frame.


\section{Artificial Swarm - Remove the Brood}

\section{Artificial Swarm - Remove the Queen}

Step 1: As soon as occupied queen cells are discovered.
Find queen and remove her, on a frame of (mostly) sealed brood + bees. Remove any queen cells from this frame after checking that there are others in the hive.
Put frame + queen in nucleus.
Add a second frame of mostly sealed brood, if wished + a frame of food + another l or 2
frames of comb (preferably) or foundation.
Shake in sufficient young worker bees to ensure that there are enough to cover the brood.
Close up the nucleus. Put green grass in the entrance if it is to remain in the same apiary.
Check through the parent colony.  Mark frames containing 2 or 3 good, unsealed, queen cells with a drawing pin.
Close up the frames in the brood box and till the remaining space with frames of comb or foundation.
Remove any sealed queen cells (Although, to use this method, the old queen must still be present so there should not be any.)
 
Step 2: 1 week later.
Go through parent colony and remove any emergency queen cells.  Best to to this three times so every other day.  The last check being a week
Select l of the cells previously marked and remove the others.
Close up the hive and leave strictly alone until queen is mated and laying.
Advantages of the method
Colony remains strong throughout.
Old queen is kept safe and is available if the new queen does not succeed.
The old queen in the nucleus quickly comes back into lay and her brood can be put back into the parent colony.
The method involves minimum time and lifting.
The nucleus is available to use for other procedures later, or can be united back to the original colony.
Disadvantages
You have to be able to find the queen easily.
The nucleus may grow very quickly, so monitor it carefully.
 
Notes from Celia Davis : SBKA Meeting March 12th 2008


\section{Artificial Swarm - Combo}



\Chapter{June}{Handle production and Unintended Swarming}

\section{Inhibit Swarming - Brood Removal}

\section{Inhibit Swarming - Moving Flying Bees}

\section{Swarming - Clipped Queen}

\section{Swarming - Unclipped Queen}

\section{Introducing a Queen}
 
 \begin{enumerate}
	\item{Best acceptance between 6 hours and 12 hours queenless or after 9 days.}
\end{enumerate}
 
 
 
\Chapter{July}{Honey Production}


 
\Chapter{August}{Honey Crop and Varroa Treatment}

Taking off the honey.

Marking and Clipping Queens


\Chapter{September}{Three colonies for winter}

\section{Cull and Merge}


\Chapter{December}{Varroa Treatment and Candy}

\section{Make Candy}




\Chapter{March}{Starvation Risk}
 
\end{document}
